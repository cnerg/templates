\section{Monte Carlo Radiation Transport}
\label{sec:bg:mc}

Blah blah blah.. background information in a section with equations and a reference to \gls{mcnp} \cite{mcnp5-theory}.
Also we used the $\backslash gls \{ \}$ command to write \gls{mcnp}.
Now we will also refer to \cref{bg:eq:fom} without having to actually type out the word ``Equation" by using $\backslash cref \{ \}$.
We can do the same for referring to \cref{sec:bg:vr} that appears below this.
Also try clicking on the section number in the last sentence.

\begin{equation}\label{bg:eq:fom}
	FOM = \frac{1}{R^2 t_{proc}}
\end{equation}

\cref{bg:eq:std} is an example using split equations.

\begin{equation}\label{bg:eq:std}
\begin{split}
	\sigma_x^2 &= \frac{\sum \left(x_i - \bar{x}\right)^2}{N-1} \\
	\sigma_{\bar{x}}^2 &= \frac{\sigma_x^2}{N}
\end{split}
\end{equation}

\subsection{Variance Reduction}
\label{sec:bg:vr}

Below is an example figure with a caption (see \cref{fig:bg:example-fig}).
Here is an example list:
\begin{enumerate}
	\item first
		\begin{itemize}
			\item sublist using bullets
			\item another item
		\end{itemize}
	\item second
\end{enumerate}

\begin{figure}[h!]
\centering
	\includegraphics[width=.3\textwidth]{./content/background/example-isosurf.png}
	\caption[Example short figure caption]{Here is a super long description and caption for this image. It is probably something we don't want to be fully written out in the list of figures at the beginning of the document. So we can use a shortened caption in square brackets in the text. It won't appear here, but that is what goes in the list.
	\label{fig:bg:example-fig}}
\end{figure}

Here is an example table to appear in list of tables (see \cref{tab:example}).
It also uses $\backslash num \{ \}$ and $\backslash SI\{ \} \{ \}$ for formatting the numbers.
The table includes a mix of standard table formatting such as the vertical bar $|$ and $\backslash hline$ and $\backslash cline$ as well as fancier formatting from booktabs package such as $\backslash toprule$, $\backslash midrule$, and $\backslash bottomrule$.
Play around with formatting to see what you like best.

\begin{table}[h!]
\centering
\caption{Example Table} \label{tab:example}
	\begin{tabular}{ c  c | c }
	\toprule
	\textbf{A} & \textbf{B} & \textbf{Some numbers} \\
	\midrule
	Joe & Smoe & \num{98.34}\% \\
	\hline
	Lisa & Pisa & \num{1e10} \\
	\hline
	Hannah & Banana & \num{0.005} \\
	\hline
	Tom & Thompson & \SI{0.5}{\centi\meter} \\
	\hline
	\multicolumn{2}{c|}{Woah merged cells!} & so cool \\
	\hline
	\multirow{2}{*}{\rotatebox{90}{XY}} & we can merge rows too & omg\\
	\cline{2-3}
	& and rotate the text! & how cool \\
	\bottomrule
\end{tabular}
\end{table}

